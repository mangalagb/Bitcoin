

%\includegraphics[scale=2]{images/parsing.eps}

There are a number of sentiments among the public regarding Bitcoin and other crypto currencies. It is not uncommon to hear "Bitcoin is a big snub to financially inept governments and corrupt banks. Bitcoin is a scam, a ponzi scheme, a bubble. Bitcoin is the future of money. Bitcoin has no future." All these statements reflect the confusion that is prevalent about the public about Bitcoin. To truly understand which of these statements is truly correct, we must look deeper into the fundamentals of Bitcoin and understand how it works, its inherent weakness and strengths. 

\section{History of Bitcoin}
In November 2008, a paper was published by Satoshi Nakamoto titled "Bitcoin: A Peer-to-Peer Electronic Cash System" \cite{nakamoto}. This paper proposes the formation of a peer-to-peer network for a new cryptocurrency. The bitcoin network came into existence in November 2009 when the first bitcoins were mined by Satoshi Nakamoto. Since then, the bitcoin network has grown and spread all over the world making it one of the most famous online currencies. It recently reached a market cap of \$1 billion. 

\section{Digital Currencies}
Bitcoin is a form of digital currency that has no physical existence but can be used to interact with the real world and buy physical goods. Transactions with these types of currencies is instantaneous and borderless. For any digital currency to be widely adopted, it must satisfy at least two basic properties:

\subsection{Central authority to keep track of transactions}
If multiple people are to use digital currencies, a central managing authority is needed to keep track of the transactions. Consider Paypal which is a popular digital currency. In order for Paypal payments to be processes, a central Paypal exchange exists which acts as a middleman between the buyer and the seller. It verifies the identities of both the parties, checks if the amount of money involved is correct or not ie. if the buyer actually has the requires money or not, and then validates the transaction. Credit cards also work in the same way. In this case, the central authority is a bank. Without this kind of an authority, a currency would fall into chaos and cheating would become common. Bitcoin does not have a central authority to validate its transactions but instead uses a peer-to-peer network to achieve the same purpose. We shall explain the details in the next section. 

\subsection{Provide safeguards against double spending}
A digital currency is in the end, a string of zeros and ones. What is to prevent one user to copy the currency and use it over and over again? This problem is called "double spending". This once again points back to the need for a central authority to prevent this kind of incident. We shall see in the next section how bitcoin solves this problem.