Bitcoins are used widely in today’s world. A lot of online e-commerce sites accept bitcoins. Bitcoins are also being increasingly accepted in bars and restaurants, hotels and other physical stores. Below are some of the current scenarios involving Bitcoin.

\section{Facilitation of e-commerce}
\subsection{Traditional e-commerce}
In a traditional environment, customers do not care about centralization and anonymity. In order for a currency to be widely used by all segments of the population, it has to be stable and reliable. Bitcoin has an ill reputation for being highly volatile with prices fluctuating from \$100 to \$1240 almost 10x changes in price versus the U.S dollar in a short period of time. Since there is no central institution handling the currency, it is highly susceptible to bad press and variance in perception of Bitcoin’s value \cite{value}. More importantly, Bitcoin has no inbuilt anti-fraud capabilities. By making transfers irreversible, it offers almost no protection to legitimate buyers or sellers. Cases of fraud involving Bitcoins are not uncommon. Due to these reasons, it is unlikely that Bitcoin can become the de facto money of the internet and replace credit card companies and other payment gateways like Paypal. 

\subsection{Micropayments}
Micropayments refer to very small payments for digital goods. The payments can be as small as 10 cents. Traditionally, there have been no efficient means to handle these kinds of payments as the transaction fees involved is too high. Bitcoin is a good competitor in this space due to its very low transaction fees. There exist services on the internet where you can tip very small amounts or "microdonate" to websites and businesses using Bitcoin \cite{value}. 

\subsection{Virtual World and Game related Commerce}
Individuals also use digital money to buy game related items such as digital clothing in Second Life or crops in Farmville \cite{farm}. At the end of 2010, around USD \$30 million transacted in Second Life was in the form of Linden Dollars(another virtual currency) \cite{linden}. As this indicates, virtual and game related markets are huge sources of revenue for vendors. Where digital currencies have not created a foothold in traditional e-commerce, they flourish in virtual e-commerce. Bitcoin has the potential to become the standard in this area. It increases trust as now the game company would not be issuing and inflating the currency as Bitcoin exists independently of any game. 

\section{Points of Failure of Bitcoin}
\subsection{External Threats}

\subsubsection{Improper Use of Discretionary Authority}
As of now, Bitcoin is unregulated by any central authority leading to its wide price fluctuation. In the future if a consortium of developers or any external authority is established to control Bitcoin’s problems, it may lead to erosion of confidence from Bitcoin even if the changes are performed with good intentions.

\subsubsection{Competing Currency}
As of now, many competitors to Bitcoins exist Namecoin, Litecoin, Zerocoin etc. They could lead to reduction of value of Bitcoin.

\subsubsection{Government Crackdown}
With Bitcoin generating a lot of bad press due its use in illicit activities, there is steady pressure from the government on it. This could lead to  crisis in conficence towards the currency.

\subsubsection{Legal Issues}
One major factor impeding Bitcoin’s use is that consumers and businesses are unsure of its exact legal standing. Traditionally, only the central government authority can issue and mint money. Existing online financial systems like credit cards and Paypal use this money. With the introduction of a new currency, it is unclear where it stands leading to Bitcoin being in the legal grey area. Unless this is resolved, a majority of businesses may be unwilling to adopt Bitcoin on a large scale.

\subsection{Technology Failures}
\subsubsection{Anonymity Failure}
Bitcoin is popular because it is considered to be anonymous by the general public. Increasingly, statistical techniques are being used to de anonymize the Bitcoin network and users. Such exposure could lead to the erosion of confidence in Bitcoin.

\subsubsection{Theft}
Bitcoins can be lost or stolen. Cases of huge thefts involving Bitcoins are common. 744,000 bitcoins were stolen from Mt.Gox in 2014. Such incidents are detrimental to the widespread adoption of Bitcoin.

\subsubsection{Denial of Service}
The Bitcoin network is susceptible to a denial of service attack. An attacker with significant computing power can flood the network and cause disruption of services. Obtaining the necessary computing power for this kind of attack is expensive but possible. It has been speculated that interested parties may include the government wishing to shut down Bitcoins or a huge group of hackers.














